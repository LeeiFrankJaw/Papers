\documentclass[a5j,12pt]{ltjtarticle}

\title{\gtfamily 談升學與選課 \\[1em]
  \hfil\normalsize\mcfamily ——給一個中校生的十二封信之七——\hspace*{-3\zw}}
\author{\small\kai 孟實}
\date{}

\usepackage{luatexja-fontspec}
\setmainjfont[TateFeatures={JFM=custom/{kaiming,vert,hwcl}}]{Adobe Song Std}
\setsansjfont{Adobe Heiti Std}
\newjfontfamily\kai{Adobe Kaiti Std}

\parindent2\zw

\usepackage{ulem}
\makeatletter
\UL@protected\def\kahasen{%
    \leavevmode \bgroup 
    \markoverwith{\lower5\p@\hbox{\sixly \char58}}\ULon}
\DeclareRobustCommand\kasen[1]{%
  \ifnum\ltjgetparameter{direction}=3\relax
    \setbox\z@\hbox{#1}\leavevmode\raise.55\zw
    \hbox to\z@{\vrule\@width\wd\z@ \@depth\z@ \@height.4\p@\hss}%
    \box\z@
  \else\underline{#1}\fi}
\makeatother

% 青 --> 靑
% 才 --> 纔
% 只 --> 祗
% 冊 --> 册

\begin{document}
\maketitle

\noindent 朋友:

你快要在中學畢業了,此時升學問題自然常在腦際盤旋。這一着也是人生一大關鍵,所以值得你慎而又慎。

升學問題分析起來便成爲兩個問題,第一是選校問題,第二是選科問題。這兩個問題自然是密切相關的,但是爲說話清晰起見,分開來說,較爲便利。

我把選校問題放在第一,因爲靑年們對於選校是最容易走入迷途的。現在\kasen{中國}社會還帶有科舉時代的資格迷。比方說小學纔畢業便希望進大學,中學纔畢業便希望出洋,出洋基本學問還沒有做好,便希望掇拾\kasen{中國}古色斑斑的東西去換博士。學校文憑只是一種找飯碗的敲門磚。學校招牌愈亮,文憑就愈行時,實學是無人過問的。社會既有這種資格迷,而資格買賣所便乘機而起。租三間鋪面,拉攏一個名流當「名譽校長」,便可掛起一個某某大學的招牌,只看\kasen{上海}一隅,大學的總數比\kasen{英}或\kasen{法}全國大學的總數似乎還要超過,誰說\kasen{中國}文化沒有提高呢?大學既多,只是稱「大學」還不能動聽,於是「大學」之上又冠以「\kasen{美國}政府註冊」的頭銜。既「大學」而又在「\kasen{美國}政府註冊」,生意自然更加茂盛了,何況許多名流又肯「熱心教育」做「名譽校長」呢?

朋友,可惜這些多如牛毛的大學都不能解決我們升學的困難,因爲那些有「名譽校長」或是「\kasen{美國}政府註冊」的大學,是預備讓有錢可花的少爺公子們去逍遙歲月的,像你我們既無錢可花,又無時光可花,只好望望然去罷。

\hfill 你的好友 \kasen{光潛}。\hspace{\parindent}

\bigskip

\hfill (原載自\kahasen{一般}雜誌第二卷第四期)\hspace{2\parindent}
\end{document}

% Local Variables:
% TeX-engine: luatex
% End:
