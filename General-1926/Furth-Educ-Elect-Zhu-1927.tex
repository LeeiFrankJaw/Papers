\documentclass[a5j,12pt]{ltjtarticle}

\title{\gtfamily 談升學與選課 \\[1em]
  \hfil\normalsize\mcfamily ——給一個中校生的十二封信之七——\hspace*{-3\zw}}
\author{\small\kai 孟實}
\date{}

\usepackage{luatexja-fontspec}
\setmainjfont[TateFeatures={JFM=custom/{kaiming,vert,hwcl}}]{Adobe Song Std}
\setsansjfont{Adobe Heiti Std}
\newjfontfamily\kai{Adobe Kaiti Std}

\parindent2\zw

\usepackage{ulem}
\makeatletter
\UL@protected\def\kahasen{%
    \leavevmode \bgroup 
    \markoverwith{\lower5\p@\hbox{\sixly \char58}}\ULon}
\DeclareRobustCommand\kasen[1]{%
  \ifnum\ltjgetparameter{direction}=3\relax
    \setbox\z@\hbox{#1}\leavevmode\raise.55\zw
    \hbox to\z@{\vrule\@width\wd\z@ \@depth\z@ \@height.4\p@\hss}%
    \box\z@
  \else\underline{#1}\fi}
\makeatother

% 青 --> 靑
% 才 --> 纔
% 只 --> 祇
% 冊 --> 册
% 卻 --> 却
% 箇 --> 個
% 真 --> 眞
% 根底 --> 根柢

% 中國 --> \kasen{中國}
% 美國 --> \kasen{美國}
% 上海 --> \kasen{上海}
% 英 --> \kasen{英}
% 法 --> \kasen{法}
% 杜威 --> \kasen{杜威}
% 哥倫比亞 --> \kasen{哥倫比亞}
% 印度 --> \kasen{印度}
% 歐美 --> \kasen{歐美}
% 亞理斯多德 --> \kasen{亞理斯多德}
% 康德 --> \kasen{康德}
% 歌德 --> \kasen{歌德}
% 亞當斯密 --> \kasen{亞當斯密}
% 英國 --> \kasen{英國}
% 羅素 --> \kasen{羅素}
% 歐 --> \kasen{歐}
% 法國 --> \kasen{法國}
% 德國 --> \kasen{德國}
% 希臘 --> \kasen{希臘}
% 光潛 --> \kasen{光潛}

\begin{document}
\maketitle

\noindent 朋友:

你快要在中學畢業了,此時升學問題自然常在腦際盤旋。這一着也是人生一大關鍵,所以值得你慎而又慎。

升學問題分析起來便成爲兩個問題,第一是選校問題,第二是選科問題。這兩個問題自然是密切相關的,但是爲說話清晰起見,分開來說,較爲便利。

我把選校問題放在第一,因爲靑年們對於選校是最容易走入迷途的。現在\kasen{中國}社會還帶有科舉時代的資格迷。比方說小學纔畢業便希望進大學,中學纔畢業便希望出洋,出洋基本學問還沒有做好,便希望掇拾\kasen{中國}古色斑斑的東西去換博士。學校文憑祇是一種找飯碗的敲門磚。學校招牌愈亮,文憑就愈行時,實學是無人過問的。社會既有這種資格迷,而資格買賣所便乘機而起。租三間鋪面,拉攏一個名流當「名譽校長」,便可掛起一個某某大學的招牌,祇看\kasen{上海}一隅,大學的總數比\kasen{英}或\kasen{法}全國大學的總數似乎還要超過,誰說\kasen{中國}文化沒有提高呢?大學既多,祇是稱「大學」還不能動聽,於是「大學」之上又冠以「\kasen{美國}政府註册」的頭銜。既「大學」而又在「\kasen{美國}政府註册」,生意自然更加茂盛了,何況許多名流又肯「熱心教育」做「名譽校長」呢?

朋友,可惜這些多如牛毛的大學都不能解決我們升學的困難,因爲那些有「名譽校長」或是「\kasen{美國}政府註册」的大學,是預備讓有錢可花的少爺公子們去逍遙歲月的,像你我們既無錢可花,又無時光可花,祇好望望然去罷。好在牠們的生意並不會因我們「杯葛」而低落的,我們求學最難得的是誠懇的良師與和愛的益友,所以選校應該以有無誠懇和愛的空氣爲準。如果能得這種學校空氣,無論是大學不是大學,我們都可以心滿意足。做學問全賴自己做,事業也全賴自己,與資格都無關係。我看過許多留學生程度不如本國大學生,許多大學生程度不如中學生,至於憑資格去混事做,學校的資格在今日是不大高貴的,你如果作此想最好去逢迎奔走,因爲那是一條較捷的路徑。

升學問題,跨進大學門限以後,還不能算完全解決。選科選課還得費你幾番躊躇。在選課的當兒,個人興趣與社會需要嘗不免互相衝突。許多人升學都以社會需要爲準。從前人都喜歡速成法政。我在中學時代,許多同學都需望進軍官學校或是教會大學。我進了高等師範,那要算是窮人末路。那時高等師範裏最時髦的是\kasen{英}文科,我選了國文科,那要算是腐儒末路。\kasen{杜威}來\kasen{中國}時,\kasen{哥倫比亞}大學的留學生們把教育學也弄得狠熱鬧。近來書店逐漸增多,出詩文集一天容易似一天,文學的風頭也算是出得十足透頂。聽說現在法政經濟又狠走時了。朋友,你是學文學或是學法政呢!「學以致用」本來不是一種壞的主張。但是資稟興趣人各不同,你假若爲社會需要而忘却自己,你就未免是一位「今之學者」了。任何科目,祇要和你興趣資稟相近,都可以發揮你的聰明才力,都可以使你效用於社會。所以你選課時,旁的問題都可以丟開,祇要問:這門功課合我的胃口麼?

我常時想,做學問,做事業,在人生中都祇能算是第二樁事。人生第一樁事是生活。我所謂「生活」是「享受」,是「領略」,是「培養生機」。假若爲學問爲事業而忘却生活,那種學問事業在人生中便失其眞正意義與價值。因此,我們不應該把自己看作社會的機械。一味迎合社會需要而不顧自己興趣的人,就沒有明白這個簡單的道理。

我把生活看作人生第一樁要事,所以不贊成早談專門;早談專門便是早走狹路,而早走狹路的人對於生活常不能見得面面俱到。前天G君對我談過一個故事,頗有趣,狠可說明我的道理。他說,有一天,一個\kasen{中國}人一個\kasen{印度}人和一位\kasen{美國}人遊歷,走到一個大瀑布前面,三人都看得發呆。\kasen{中國}人說,『自然眞是美麗!』\kasen{印度}人說,『在這種地方纔見到神的力量哩!』\kasen{美國}人說,『可惜喏水力都空費了!』這三句話各各不同,各有各的眞理,各也有各的缺陷。在完美的世界裏,我們在瀑布中應能同時見到自然的美麗,神力的廣大和水力的實用。許多人因爲站在狹路上,祇能見到諸方面的某一面,便說他人所見到的都不如他的眞確。前幾年大家曾像煞有介事地爭辯哲學和科學,爭辯美術和宗教,不都是坐井觀天誣天渺小麽?

我最怕和談專門的書呆子在一起,你同他談話,他三句話就不離本行。談到本行以外,旁人所以爲興味盎然的事物,他聽之則麻木不能感覺。像這樣的人是因爲做學問而忘記生活了。我特地提出這一點來說,因爲我想現在許多人大談職業教育,而不知單講職業教育也頗危險。我並非反對職業教育,我却深深地感覺到職業教育應該有寬大自由教育(Liberal Education)做根柢。倘若先沒有多方面的寬大自由教育做根柢,則職業教育的流弊,在個人方面,常使生活單調乏味,在社會方面,常使文化浮淺褊狹。

許多人一開口就談專門(Specialise),就談研究(Research work)。他們說,\kasen{歐美}學問進步所以迅速,由於治學尚專門。原來不專則不精,固是自然之理。可是「專」也並非是任何人所能說的。倘若基礎樹得不寬廣,你就是「專」,也決不能「專」到多遠路。自然和學問都是有機的系統,其中各部分常息息相通,牽此則動彼。倘若你對於其他各部分都茫無所知,而專門研究某一部分,實在是不可能的。哲學和歷史,須有一切學問做根柢。文學與哲學歷史也密切相關。科學是比較可以專習的,而實亦不盡然。比方生物學,要研究到精深的地步,不能不通化學,不能不通物理學,不能不通地質學,不能不通數學和統計學,不能不通心理學。許多人連動物學和植物學的基礎也沒有,便談專門研究生物學,是無異於未學爬而先學跑的。我常時想,學問這件東西,先要能博大而後能精深。「博學守約」,眞是至理名言。\kasen{亞理斯多德}是種種學問的祖宗。\kasen{康德}在大學裏幾乎能擔任一切功課的教授。\kasen{歌德}蓋代文豪而於科學上也狠有建樹。\kasen{亞當斯密}是\kasen{英國}經濟學的始祖,而他在大學是教授文學的。近如\kasen{羅素},他對於數學,哲學,政治學樣樣都能登峯造極。這是我信筆寫來的幾個確例。西方大學者(尤其是在文學方面)大半都能同時擅長几種學問的。

我從前預備再做學生時,也曾癡心妄想過專門研究某科中某某問題。來\kasen{歐}以後,看看旁人做學問所走的路徑,纔覺悟像我這樣淺薄,就談專門研究,眞可謂「顏之厚矣」!我此時纔知道從前在國內聽大家所常談的「專門」是怎麼一回事。\kasen{中國}一般學者的通弊就在不重根基而侈談高遠。比方講「東西文化」的人,可以不通哲學,可以不通文學和美術,可以不通歷史,可以不通科學,可以不懂宗教,而信口開河,憑空立說;歷史學者聞之竊笑,科學家聞之竊笑,文藝批評學者聞之竊笑,祇是發議論者自己在那裏洋洋得意。再比方著世界文學史的人,\kasen{法國}文學可以不懂,\kasen{英國}文學可以不懂,\kasen{德國}文學可以不懂,\kasen{希臘}文學可以不懂,\kasen{中國}文學可以不懂,而東抄西襲,堆砌成篇,使\kasen{法國}文學學者見之竊笑,\kasen{英國}文學學者見之竊笑,\kasen{中國}文學學者見之竊笑,祇是著書人自己在那裏大吹喇叭。這眞所謂「放屁放屁,眞正豈有此理」!

朋友,你就是升到大學裏去,千萬莫要染着時下習氣,侈談高遠而不注意把根基打得寬大穩固。我和你相知甚深,客氣話似用不着說。我以爲你在中學所打的基本學問的基礎還不能算是穩固,還不能使你進一步談高深專門的學問。至少在大學頭一二年中,你須得盡力多選功課,所謂多選功課,自然也有一個限制。貪多而不務得,也是一種毛病。我是說,在你的精力時間可能範圍以內,你須極力求多方面的發展。

最後,我這番話祇是針對你的情形而發的。我不敢說一切中學生都要趁着這條路走。但是對於預備將來專門學某一科而謀深造的人,──尤其是所學的關於文哲和社會科學方面,──我的忠告總含有若干眞理。

同時,我也狠願聽聽你自己的意見。

\hfill 你的好友 \kasen{光潛}。\hspace{\parindent}

\bigskip

\hfill (原載自\kahasen{一般}雜誌第二卷第四期〔一九二七年四月號〕)\hspace{2\parindent}
\end{document}

% Local Variables:
% TeX-engine: luatex
% End:
