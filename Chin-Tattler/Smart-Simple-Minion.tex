\documentclass[a5j,12pt]{ltjtarticle}

\title{\gtfamily 聰明人和傻子和奴才 \\[1em]
\hfil\normalsize\mcfamily ——野草之二十\hspace*{-3\zw}}
\author{\small\kai 魯迅}
\date{}

\newif\ifadobe
% \adobetrue

\usepackage{luatexja-fontspec}
\ifadobe
  \setmainjfont[
    TateFeatures={JFM=custom/{quanjiao,vert,otf};+cpct},
    AltFont={{Range="FF1A-"FF1B,Font=Adobe Song Std}}
  ]{FZNew XiuLiB-Z11}
\else
  \setmainjfont[TateFeatures={JFM=custom/{quanjiao,vert}}]{FZNew XiuLiB-Z11}
\fi
\setsansjfont{FZHeiB-B01}
\newjfontfamily\kai{FZKaiB-Z03}

\parindent2\zw

\newcommand*{\vcolon}{\ifadobe:\else\hbox{\yoko:}\fi}
\newcommand*{\vscolon}{\ifadobe;\else\hbox{\yoko;}\fi}

\begin{document}
\maketitle

奴才總不過是尋人訴苦。只要這樣,也只能這樣。有一日,他遇到一個聰明人。

「先生!」他悲哀地說,眼淚聯成一線,就從眼角上直流下來。「你知道的。我所過的簡直不是人的生活。吃的是一天未必有一餐,這一餐又不過是高粱皮,連豬狗都不要吃的,尚且只有一小碗……。」

「這實在令人同情。」聰明人也慘然說。

「可不是麼!」他高興了。「可是做工是晝夜無休息的\vcolon{}清早擔水晚燒飯,上午跑街夜磨麵,晴洗衣裳雨張傘,冬燒汽鑪夏打扇。半夜要煨銀耳,侍候主人耍錢\vscolon{}頭錢從來沒分,有時還挨皮鞭……。」

「唉唉……。」聰明人嘆息著,眼圈有些發紅,似乎要下淚。

「先生!我這樣是敷衍不下去的。我總得另外想法子。可是什麼法子呢?……」

「我想,你總會好起來……。」

「是麼?但願如此。可是我對先生訴了冤苦,又得了你的同情和慰安,已經舒坦得不少了。可見天理沒有滅絕……。」

\bigskip

但是,不幾日,他又不平起來了,仍然尋人去訴苦。

「先生!」他流著淚說,「你知道的。我住的簡直比豬窩還不如。主人並不將我當人\vscolon{}他對他的叭兒狗還要好到幾萬倍……。」

「混帳!」那人大叫起來\vscolon{}使他吃驚了。那人是一個傻子。

「先生,我住的只是一間破小屋。又溼,又陰,滿是臭蟲,睡下去就咬得真可以。穢氣衝著鼻子,四面又沒有一個窗……。」

「你不會要你的主人開一個窗的麼?」

「這怎麼行?……」

「那麼,你帶我去看去!」

傻子跟奴才到他屋外,動手就砸那泥牆。

「先生!你幹什麼?」他大驚地說。

「我給你打開一個窗洞來。」

「這不行!主人要罵的!」

「管他呢!」他仍然砸。

「人來呀!強盜在毀咱們的屋子了!快來呀!遲一點可要打出窟窿來了!……」他哭嚷著,在地上團團地打滾。

一群奴才都出來了,將傻子趕走。

聽到了喊聲,慢慢地最後出來的是主人。

「有強盜要來毀咱們的屋子,我首先叫喊起來,大家一同把他趕走了。」他恭敬而得勝地說。

「你不錯。」主人這樣誇獎他。

\bigskip

這一天就來了許多慰問的人,聰明人也在內。

「先生,這回因為我有功,主人誇獎了我了。你先前說我總會好起來,實在是有先見之明……。」他大有希望似的高興地說。

「可不是麼……。」聰明人也代為高興似的回答他。

\bigskip

\hspace{\stretch{2}} \date{一九二五年十二月二十六日。}\hspace{\stretch{1}}
\end{document}

% Local Variables:
% TeX-engine: luatex
% End:
